%\documentclass[a4paper,12pt]{article}
\documentclass[twocolumn]{article}
\setlength\columnsep{25pt}
\usepackage{mathtools}
\begin{document}


\title{EE5600 Assignment 1}
\author{Ealleti Sai Vikram [EE20MTECH11006]}
\date{\today}
\maketitle

\textbf{\emph{Abstract} - This document contains the \\solution to Lines and planes problem}


% comment declaration

%section declaration
\section*{Problem:}
Find the equation of the plane passing through
the line of intersection of the planes
\\   
\hspace*{2cm}  
$\begin{pmatrix}1&1&1\end{pmatrix}$$\cdot$X = 1  and
\\
\hspace*{2cm}
$\begin{pmatrix}2&3&-1\end{pmatrix}$$\cdot$X = 4
\\
and parallel to X-axis.
%another section
\section*{Solution:}
For the above problem we try to find equation
\\ of plane ax + by + cz = d using point-normal approach.
The process is as follows :
\\
\\\textbf{Step 1}: Finding the normal vector of plane for finding coefficients of plane equation (a,b,c).
\\
\\ For this we utilize the concept of cross product of normal vectors of two planes.
\\\hspace*{0.5cm} Normal vector of plane 1:$\vec{N_{p1}}$= $\hat{i}$  + $\hat{j}$  + $\hat{k}$
\\\hspace*{0.5cm} Normal vector of plane 2:$\vec{N_{p2}}$= 2$\hat{i}$ + 3$\hat{j}$ - $\hat{k}$
\\Now cross-product of $\vec{N_{p1}}$ and $\vec{N_{p2}}$ yields
\\\hspace*{1.5cm}$\vec{N_{p1}}$$\times$$\vec{N_{p2}}$ = -4$\hat{i}$ + 3$\hat{j}$ + 1$\hat{k}$
\\Call the resultant vector  $\vec{P_{p3}}$
\\\hspace*{1.5cm}$\vec{P_{p3}}$ = -4$\hat{i}$ + 3$\hat{j}$ + 1$\hat{k}$
\\This Vector is perpendicular to $\vec{N_{p1}}$ and $\vec{N_{p2}}$ ,hence it gives the direction of intersecting line  which is present in required plane. 
\\Now to find out normal vector we use the concept that dot product of orthogonal vectors gives 0.
\\
\\Normal vector of required plane has form b$\hat{j}$+c$\hat{k}$ as plane is parallel to X-axis. Call the vector $\vec{N_{p3}}$.
\\Dot-product of $\vec{P_{p3}}$ and $\vec{N_{p3}}$ gives
\\\hspace*{1.5cm}$\vec{P_{p3}}$ $\cdot$ $\vec{N_{p3}}$ =  3b + k = 0
\\\hspace*{2.65cm} k = -3b
\\Therefore the coefficients of plane equation (a,b,c)=(0,1,-3)
\\
\\\textbf{Step 2}: Finding the point $(x_{0},y_{0},z_{0})$ which lies on required plane to complete the plane equation.
\\
\\For finding a point lets substitute x=0 in plane equations of plane 1 and 2 and try to solve for y and z coordinates of point.
\\\hspace*{1.5cm}X=$\begin{pmatrix}x\\y\\z\end{pmatrix}$=$\begin{pmatrix}0\\y\\z\end{pmatrix}$
\\$\begin{pmatrix}1&1&1\end{pmatrix}$$\cdot$X = 1 $\implies$ y + z = 1     $ \hspace*{1.26cm}$  --(1)
\\$\begin{pmatrix}2&3&-1\end{pmatrix}$$\cdot$X = 1 $\implies$ 3y - z =-4  $ \hspace*{1cm}$ --(2)
\\Solving equations (1) and (2) we get
\\\hspace*{1.5cm} y=0.75 and z=-1.75
\\Therefore point in plane
\\\hspace*{1.2cm}$(x_{0},y_{0},z_{0})$ = (0,-0.75,1.75)
\\Now coefficient d= $ a \cdot x_{0} + b \cdot y_{0} + c \cdot z_{0}$
\\\hspace*{2.57cm} = -1*0.75 - 3*1.75 = -6
\\The required plane equation is y -3z = -6
\\It can also be represented as $\begin{pmatrix}0&1&-3\end{pmatrix}$$\cdot$X = -6
\subsection{}
Here ,this is called a subsection.
%another subsection
\subsection{}
Another subsection.

\section{Labelling}
\label{sec2}
This section can be accessed to other parts of document. Following is the implementation of it...
\section{Implementation}
Referring to  section \ref{sec2} on page \pageref{sec2}


	


\end{document}












