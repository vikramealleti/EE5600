%\documentclass[a4paper,12pt]{article}
\documentclass[twocolumn]{article}
\setlength\columnsep{25pt}
\usepackage{mathtools}
\usepackage{esvect}
\usepackage{hyperref}
\usepackage[utf8]{inputenc}
\usepackage{amsmath}
\usepackage{listings}
\usepackage{physics}
\newcommand{\myvec}[1]{\ensuremath{\begin{pmatrix}#1\end{pmatrix}}}
\makeatletter
\renewcommand*\env@matrix[1][*\c@MaxMatrixCols c]{%
   \hskip -\arraycolsep
   \let\@ifnextchar\new@ifnextchar
   \array{#1}}
\makeatother
\let\vec\mathbf
\newcommand*{\matminus}{%
  \leavevmode
  \hphantom{0}%
  \llap{%
    \settowidth{\dimen0 }{$0$}%
    \resizebox{1.1\dimen0 }{\height}{$-$}%
  }%
}
\begin{document}
\title{EE5600 Assignment 1}
\author{Ealleti Sai Vikram [EE20MTECH11006]}
\date{\today}
\maketitle



\textbf{\emph{Abstract} - This document contains the \\solution to Lines and planes problem}


% comment declaration

%section declaration
\section*{Problem:}
Find the equation of the plane passing through
the line of intersection of the planes
\begin{equation}
\begin{pmatrix}{\bf1}&{\bf1}&{\bf1}\end{pmatrix}\cdot{\bf \vec{X}} ={\bf 1}, \nonumber
\end{equation}
\begin{equation}
\begin{pmatrix}{\bf2}&{\bf 3}&{\bf -1}\end{pmatrix}\cdot{\bf \vec{X}} ={\bf -4} \nonumber
\end{equation}
and parallel to X-axis.

\section*{Solution:}

Equation of plane 1 :
\begin{equation}\begin{pmatrix}{\bf1}&{\bf1}&{\bf1}\end{pmatrix}\cdot{\bf \vec{X}} = \bf1   \label{eqn1}
\end{equation}
Equation of plane 2 :
\begin{equation}\begin{pmatrix}{\bf2}&{\bf 3}&{\bf -1}\end{pmatrix}\cdot {\bf \vec{X}} = \bf-4 \label{eqn2}
\end{equation}

Let the equation of plane 3 which passes through line made by intersection of planes 1 and 2 and being parallel to X-axis : 
\begin{equation}
\begin{pmatrix}{\bf0}&{\bf p}&{\bf q}\end{pmatrix}\cdot{\bf \vec{X}} = \bf c \label{eqn3}
\end{equation}
Now if three planes are passing through same line Then the Echelon matrix form obtained must be of form :
\begin{equation}
\begin{pmatrix}[ccc|c]
 x & x & x & x\\  0 & x & x & x \\ 0 & 0 & 0 & 0
\end{pmatrix} \nonumber
\end{equation}
\\The augumented matrix from three planes :
\begin{equation}
\begin{pmatrix}[ccc|c]
 1 & 1 & 1 & 1\\  2 & 3 & \matminus1 & \matminus4 \\ 0 & p & q & c \label{eqn4}
\end{pmatrix} 
\end{equation}
\\ Performing the following row operations on
\eqref{eqn4}
\begin{align}
r_2 \rightarrow r_2-2\cdot r_1  \label{eq5} \\
r_3 \rightarrow r_3-p\cdot r_2 \label{eq6}
\end{align}
we end up with
\begin{equation}
\begin{pmatrix}[ccc|c]
 1 & 1 & 1 & 1\\  0 & 1 & \matminus3 & \matminus6 \\ 0 & 0 & q +3\cdot p & c +6\cdot p
\end{pmatrix} \label{eq7}
\end{equation}
\\To convert this matrix to Echelon form for three planes passing through same line the last row must be made zeroes.
\begin{align}
\bf \Longrightarrow q=\matminus3p \hspace*{0.5cm}\&\hspace*{0.5cm} c=\matminus6p \label{eqn8}
\end{align}
Therefore the required plane equation :
\begin{align}
\begin{pmatrix}{\bf0}&{\bf p}&{\bf \matminus 3p }\end{pmatrix}\cdot{\bf \vec{X}} = \bf \matminus6p \label{eqn9}
\end{align}
Normalizing  with {\bf `p'} our plane equation becomes,
\begin{align}
\begin{pmatrix}{\bf0}&{\bf 1}&{\bf \matminus 3 }\end{pmatrix}\cdot{\bf \vec{X}} = \bf \matminus6 \label{eqn10}
\end{align}
\subsection*{Link For Python Code}
\href{https://github.com/vikramealleti/EE5600/blob/master/Assignment_1/Linalgassignment.py}{code.py}
\end{document}

